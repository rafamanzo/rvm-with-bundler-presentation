
\section{Lidando com múltiplas versões de tudo}

\frame{
  \begin{center}
    \LARGE Lidando com múltiplas versões de tudo
  \end{center}
}

\frame{
  \frametitle{rbenv vs. RVM}
  \pause
  \begin{columns}
    \column[c]{.5\textwidth}
      \textbf{rbenv}
      \begin{itemize}
        \item Usa \texttt{\$PATH}
        \item Sem bash hacking
        \item Sem cofig file
        \item Não instala Ruby
        \item Gemsets como plugin
      \end{itemize}
    \column[c]{.5\textwidth}
      \textbf{RVM}
      \begin{itemize}
        \item Usa \textit{login shell}
        \item Sobrescreve commandos shell
        \item Com config file
        \item Instala Ruby (ver adiante)
        \item Gemsets out-of-the-box
      \end{itemize}
  \end{columns}
}

\subsection{RVM}

\frame{
  \frametitle{Instalação single-user}
  \pause
  Instala o RVM na \texttt{HOME} do usuário atual com dois comandos:
  \begin{enumerate}
    \pause
    \item Instale a chave pública GPG:
          \begin{block}{bash}
            \texttt{\$ gpg --keyserver hkp://keys.gnupg.net --recv-keys 409B6...}
          \end{block}
    \pause
    \item Execute o script de instalação:
          \begin{block}{bash}
            \texttt{\$ \textbackslash{}curl -sSL https://get.rvm.io | bash -s
            stable}
          \end{block}
  \end{enumerate}

  \pause
  \vspace{1em}
  Erros comuns:
  \begin{itemize}
    \item Não verificar o \texttt{.profile}
    \item Não ativar \textit{login shell}
  \end{itemize}

  \pause
  \vspace{1em}
  Para mais informações veja \url{https://rvm.io/rvm/install}
}

\frame{
  \frametitle{Instalando uma versão de ruby}
  \pause
  Com o RVM pronto, você pode instalar a versão desejada de Ruby:

  \pause
  \begin{block}{bash}
    \texttt{\$ rvm install 2.1.1}
  \end{block}

  \pause
  Para saber as versões disponíveis:

  \pause
  \begin{block}{bash}
    \texttt{\$ rvm list known}
  \end{block}

  \pause
  Para mudar entre \textit{patches}:

  \pause
  \begin{block}{bash}
    \texttt{\$ rvm ruby 2.1.1-p451}
  \end{block}
}

\frame{
  \frametitle{Gemsets}
  \pause
  Ambientes isolados de ruby e suas gems
  
  \begin{itemize}
    \pause
    \item Cada Gemset pode usar todas as versões de ruby instaladas
    \pause
    \item Você pode mudar entre Gemsets à vontade
    \pause
    \item O RVM vem com uma Gemset global por padrão
  \end{itemize}

  \pause
  Criando uma Gemset:

  \begin{block}{bash}
    \texttt{\$ rvm gemset create myproject \\
            Gemset 'myproject' created}
  \end{block}

  \pause
  Usando uma Gemset:

  \begin{block}{bash}
    \texttt{\$ rvm use 2.1.1@myproject}
  \end{block}
}

\frame{
  \frametitle{Dotfiles}
}

\frame{
  \frametitle{Comandos úteis}
}

